\documentclass{article}
\usepackage[T1]{fontenc}
\usepackage{caption}
\usepackage[htt]{hyphenat}
\usepackage{multicol}
\usepackage{booktabs}
\usepackage{float}
\usepackage{commath}
\usepackage{pgfplots}
\usepackage{amsmath}
\usepackage{listings}
\usepackage{bm}
\usepackage{xcolor}
% \usepackage[euler]{textgreek}
\usepackage{amsfonts}
\usepackage{amssymb}
\usepackage{syntax}
\usepackage{enumitem}
\usepackage [english]{babel}
\usepackage [autostyle, english = american]{csquotes}
\MakeOuterQuote{"}
\usepackage[linguistics]{forest}
\usetikzlibrary{trees}

\definecolor{mygreen}{RGB}{28,172,0} % color values Red, Green, Blue
\definecolor{mylilas}{RGB}{170,55,241}
%-------------------------------------------
\setlength{\textwidth}{7.0in}
\setlength{\oddsidemargin}{-0.35in}
\setlength{\topmargin}{-0.5in}
\setlength{\textheight}{9.0in}
\setlength{\parindent}{0.3in}
\title{IE 306 - Homework 1 \\ Call Centre Simulation}
\author{Mehdi Saffar - 2016400411 \\ Burak Berk Özer - 2016400015 \\ Mehmet Umut Öksüz - 2016400096}

%=== DOCUMENT
\begin{document}
\maketitle

\section{Introduction}
In this project, a call center is being simulated. There is a pipeline in place for customers to get their services.

First of all, a customer should reach out to a voice recognition system. Voice recognition systems gathers
information about the caller and connects it to one of two operators. Each operator have their own queues. After
customers waits an amount of time in a queue, s/he gets served by the operator.

In each step of the simulation, a distribution function is being utilized to make decision. For example, the time a
customer spends in voice recognition system is independently exponentially distributed with mean 5 minutes.

Also, there is a number of failure points in this system. A customer might not be able to reach to voice recognition
system due to overcapacity, get connected to a wrong operator or s/he might spend too much time in operator queue and
reneged.

After taking all these into consideration, these program simulates the call center and derives important statistics
about the system. For example, utilization of the answering system, average total waiting time. These statistics
could be used to impact the future of the call center and provide better service for all parties.

\section{Simulation Logic}
In our implementation, we have the object world which is instantiated at the beginning of the each simulation. World
object firstly initializes our call center operators that are inherited from \texttt{simpy.Resource}. The automated
answering system is also initialized here with a capacity of 100. After these steps, run function of the world is
executed. This function defines 3 simpy processes to trigger different generators. First generator generates
customers using random time intervals sampled from given distribution. Other two generators generate breaks for
operators. These created customer and break objects execute events and progress the simulation that continues until
desired amount of successful calls are completed.

Callers are represented by a \texttt{Customer} object. When a customer instance is created, it calls its 'call'
function to start calling process. In call function, it is firstly checked that if the answering system is full. If
it is full, then customer cannot continue, he/she is directly kicked out of the system. If customer can connect to an
answering system then they spend a pre-determined duration of time which is generated from an exponential
distribution with mean 5. Then the operator to which that customer should be routed to is decided using a uniform
distribution. After selecting the operator, we randomly decide if this routing was wrong or not. If a wrong routing
has occured, customer immediately exits the system. Otherwise, customer enters the queue of its
operator. In code, it requests the operator object that is a \texttt{simpy.Resource}. He/she waits for the operator.
But if a timeout of 10 minutes occurs earlier the connection to the operator, the customer exits the queue and we
collect its statistics before kicking him/her out of the system.

After connecting to operator, customer spends a
certain time randomly determined using given distribution. Operator is released and can serve a new customer from the
queue after that point. We check if we have enough answered customers after all successfull calls. Simulation is
continued unless we reach that number.

Since the number of breaks are Poisson distributed, we can use exponential distribution to determine the interval of
time between breaks. So, in average, one break per hour is generated. Using an exponential distribution of mean 60,
we create breaks for each shift. We also know when a break is created so, using that information, we know that if a
break is created in the previous shift and we can discard it from the queue. To prevent an operator from taking a
break when it has customers waiting in the queue, we assign lower priority to breaks. So, breaks wait in the queue
until no customer exist in the queue.

\section{Simulation Output}
We simulate the two sets of 10 runs with the same seeds with 1000 and 5000 answered calls.

\subsection{Simulating 1000 answered calls}
\begin{table}[H]
    \tiny
    \centering
    \caption{Simulating 10 runs of 1000 answered calls}
    \begin{tabular}{@{}llllllllll@{}}
        \toprule
        \textbf{Run} & \textbf{Seed} & \textbf{Sys. Util} & \textbf{Op1 Util} & \textbf{Op2 Util} & \textbf{Avg. Wait} & \textbf{W/S} & \textbf{Avg. \# in Op1 Q} & \textbf{Avg. \# in Op2 Q} & \textbf{\# Unsatisfied} \\ \midrule
        1            & 37            & 0.008              & 0.437             & 0.415             & 2.147              & 0.161        & 0.142                     & 0.172                     & 172                     \\
        2            & 12            & 0.008              & 0.447             & 0.419             & 2.25               & 0.169        & 0.157                     & 0.189                     & 197                     \\
        3            & 72            & 0.008              & 0.444             & 0.405             & 2.03               & 0.155        & 0.149                     & 0.153                     & 184                     \\
        4            & 9             & 0.008              & 0.446             & 0.393             & 2.061              & 0.156        & 0.153                     & 0.149                     & 202                     \\
        5            & 75            & 0.009              & 0.462             & 0.408             & 2.441              & 0.176        & 0.161                     & 0.209                     & 202                     \\
        6            & 5             & 0.008              & 0.431             & 0.401             & 2.072              & 0.153        & 0.141                     & 0.16                      & 186                     \\
        7            & 79            & 0.008              & 0.411             & 0.4               & 2.14               & 0.162        & 0.132                     & 0.179                     & 173                     \\
        8            & 64            & 0.008              & 0.489             & 0.403             & 1.997              & 0.149        & 0.151                     & 0.144                     & 176                     \\
        9            & 16            & 0.008              & 0.419             & 0.423             & 2.012              & 0.154        & 0.131                     & 0.171                     & 180                     \\
        10           & 1             & 0.008              & 0.476             & 0.439             & 2.282              & 0.168        & 0.152                     & 0.201                     & 175                     \\ \bottomrule
    \end{tabular}
\end{table}

\begin{table}[H]
    \tiny
    \centering
    \caption{Statistical analysis of the 10 runs of 1000 answered calls}
    \begin{tabular}{@{}lllllllllll@{}}
        \toprule
        \textbf{} & \textbf{Sys. Util} & \textbf{Op1 Util} & \textbf{Op2 Util} & \textbf{Avg. Wait} & \textbf{W/S} & \textbf{Avg. \# in Op1 Q} & \textbf{Avg. \# in Op2 Q} & \textbf{\# Unsatisfied} \\ \midrule
        mean      & 0.008              & 0.446             & 0.411             & 2.143              & 0.16         & 0.147                     & 0.173                     & 184.7                   \\
        std       & 0.0                & 0.024             & 0.014             & 0.142              & 0.009        & 0.01                      & 0.022                     & 11.748                  \\
        min       & 0.008              & 0.411             & 0.393             & 1.997              & 0.149        & 0.131                     & 0.144                     & 172.0                   \\
        25\%      & 0.008              & 0.432             & 0.402             & 2.038              & 0.154        & 0.141                     & 0.155                     & 175.25                  \\
        50\%      & 0.008              & 0.445             & 0.406             & 2.106              & 0.158        & 0.15                      & 0.172                     & 182.0                   \\
        75\%      & 0.008              & 0.458             & 0.418             & 2.224              & 0.166        & 0.153                     & 0.186                     & 194.25                  \\
        max       & 0.009              & 0.489             & 0.439             & 2.441              & 0.176        & 0.161                     & 0.209                     & 202.0                   \\ \bottomrule
    \end{tabular}
\end{table}

\subsection{Simulating 5000 answered calls}

\begin{table}[H]
    \tiny
    \centering
    \caption{Simulating 10 runs of 5000 answered calls}
    \begin{tabular}{@{}llllllllll@{}}
        \toprule
        \textbf{Run} & \textbf{Seed} & \textbf{Sys. Util} & \textbf{Op1 Util} & \textbf{Op2 Util} & \textbf{Avg. Wait} & \textbf{W/S} & \textbf{Avg. \# in Op1 Q} & \textbf{Avg. \# in Op2 Q} & \textbf{\# Unsatisfied} \\ \midrule
        1            & 37            & 0.008              & 0.45              & 0.4               & 2.112              & 0.159        & 0.148                     & 0.164                     & 904                     \\
        2            & 12            & 0.008              & 0.433             & 0.415             & 2.088              & 0.157        & 0.138                     & 0.176                     & 944                     \\
        3            & 72            & 0.008              & 0.437             & 0.411             & 1.983              & 0.151        & 0.131                     & 0.163                     & 869                     \\
        4            & 9             & 0.008              & 0.447             & 0.397             & 2.078              & 0.155        & 0.145                     & 0.162                     & 995                     \\
        5            & 75            & 0.009              & 0.44              & 0.422             & 2.067              & 0.154        & 0.135                     & 0.179                     & 919                     \\
        6            & 5             & 0.009              & 0.453             & 0.423             & 2.22               & 0.163        & 0.147                     & 0.19                      & 919                     \\
        7            & 79            & 0.008              & 0.441             & 0.409             & 2.098              & 0.157        & 0.143                     & 0.171                     & 937                     \\
        8            & 64            & 0.008              & 0.453             & 0.396             & 2.037              & 0.152        & 0.142                     & 0.155                     & 931                     \\
        9            & 16            & 0.008              & 0.422             & 0.421             & 2.073              & 0.157        & 0.133                     & 0.175                     & 931                     \\
        10           & 1             & 0.008              & 0.454             & 0.41              & 2.084              & 0.154        & 0.143                     & 0.165                     & 933                     \\ \bottomrule
    \end{tabular}
\end{table}

\begin{table}[H]
    \tiny
    \centering
    \caption{Statistical analysis of the 10 runs of 5000 answered calls}
    \begin{tabular}{@{}lllllllllll@{}}
        \toprule
        \textbf{} & \textbf{Sys. Util} & \textbf{Op1 Util} & \textbf{Op2 Util} & \textbf{Avg. Wait} & \textbf{W/S} & \textbf{Avg. \# in Op1 Q} & \textbf{Avg. \# in Op2 Q} & \textbf{\# Unsatisfied} \\ \midrule
        mean      & 0.008              & 0.443             & 0.41              & 2.084              & 0.156        & 0.14                      & 0.17                      & 928.2                   \\
        std       & 0.0                & 0.01              & 0.01              & 0.06               & 0.004        & 0.006                     & 0.01                      & 31.776                  \\
        min       & 0.008              & 0.422             & 0.396             & 1.983              & 0.151        & 0.131                     & 0.155                     & 869.0                   \\
        25\%      & 0.008              & 0.438             & 0.402             & 2.068              & 0.154        & 0.136                     & 0.163                     & 919.0                   \\
        50\%      & 0.008              & 0.444             & 0.41              & 2.081              & 0.156        & 0.142                     & 0.168                     & 931.0                   \\
        75\%      & 0.008              & 0.452             & 0.42              & 2.096              & 0.157        & 0.144                     & 0.176                     & 936.0                   \\
        max       & 0.009              & 0.454             & 0.423             & 2.22               & 0.163        & 0.148                     & 0.19                      & 995.0                   \\ \bottomrule
    \end{tabular}
\end{table}

\subsection{Summary of both sets}

\begin{table}[H]
    \centering
    \tiny
    \caption{Statistical analysis of all the 20 runs of 1000 and 5000 answered calls}
    \begin{tabular}{@{}lllllllllll@{}}
        \toprule
        \textbf{} & \textbf{Sys. Util} & \textbf{Op1 Util} & \textbf{Op2 Util} & \textbf{Avg. Wait} & \textbf{W/S} & \textbf{Avg. \# in Op1 Q} & \textbf{Avg. \# in Op2 Q} & \textbf{\# Unsatisfied} \\ \midrule
        mean      & 0.008              & 0.445             & 0.41              & 2.114              & 0.158        & 0.144                     & 0.171                     & 556.45                  \\
        std       & 0.0                & 0.018             & 0.012             & 0.111              & 0.007        & 0.009                     & 0.017                     & 382.12                  \\
        min       & 0.008              & 0.411             & 0.393             & 1.983              & 0.149        & 0.131                     & 0.144                     & 172.0                   \\
        25\%      & 0.008              & 0.436             & 0.401             & 2.055              & 0.154        & 0.137                     & 0.162                     & 183.0                   \\
        50\%      & 0.008              & 0.445             & 0.41              & 2.081              & 0.156        & 0.143                     & 0.171                     & 535.5                   \\
        75\%      & 0.008              & 0.453             & 0.42              & 2.142              & 0.161        & 0.15                      & 0.179                     & 931.0                   \\
        max       & 0.009              & 0.489             & 0.439             & 2.441              & 0.176        & 0.161                     & 0.209                     & 995.0                   \\ \bottomrule
    \end{tabular}
\end{table}


\section{Interpretation of results}

\begin{itemize}
    \item Answering system is not utilized enough (only $0.008$). The call centre can reduce the number of channels down to around $0.10 * 100 = 10$ channels without affecting the overall performance.
    \item Since the answering system sends with $0.9 * 0.7$ probability to operator 1, we would expect that operator 2 would be more utilized than operator 1 but we can see that, on average, operator 1 is more utilized than operator 2. This can be explained by the difference of distribution of their service time, where operator 1 has a higher mean.
    \item We see that both operators' utilization is not optimal since they only half of the time. This might be improved by making the automated answering system less erroneous and/or giving less breaks.
    \item The average waiting time is 2 minutes, which is reasonable since it only represents $15.8\%$ of the time spent during the call.
    \item The queue for both operators is most of the time empty which is good.
\end{itemize}

\section{Conclusion}
This project demonstrates the use of SimPy and how it can be leveraged to understand the dynamics of the queuing
network of a call centre. SimPy was very flexible and easy to learn. Although the system is not too complex, some
edge cases were tricky to solve. Overall, this project helped us get better insights on how simulation are run and
gave us an opportunity of making our own statistical analysis.
\end{document}
